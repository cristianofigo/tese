\section{A essência da fusão e do contraste}

Para haver fusão entre as escrituras instrumental e eletroacústicas,
será necessário que haja \textit{transferências localizadas} de 
características espectrais de uma esfera de atuação à outra.
Aquilo que se funde com outra coisa, assim o faz pela \textit{similaridade
absoluta}, com esta outra coisa, de ao menos um aspecto de sua 
constituição. Nesse sentido, tratando-se de sons eletroacústicos 
pré-elaborados em estúdio, a eleição do material constitutivo de partida
adquire grande relevância: será mais plausível trabalhar, sobre suporte,
com sons oriundos dos próprios instrumentos do que com proveniências
díspares, sem qualquer relação de origem com a materialidade corpórea
dos instrumentos utilizados. Ainda que as transformações em curso possam ser
bem drásticas, o uso de material constitutivo similar faz com que
haja preponderância em conservar algum aspecto energético que confira
identidade às texturas sonoras resultantes.
....
Como quer que seja, na fusão instaura-se uma condição de \textit{dúvida}.
Em certa medida, fusão implica propositadamente, da parte do compositor,
\textit{confusão} para o ouvinte....o ouvinte recai em constantes dúvidas
acerca da natureza daquilo que se ouve: se advém do instrumento ou da 
emissão eletroacústica, se se opera ao vivo uma dinamização espacial, harmônica
, tímbrica e temporal da escritura instrumental ou se será defronte de
estruturas pré-elaboradas em estúdio, constituidas a partir dos próprios
instrumentos ou a estes timbricamente correlatas. Em relação a proveniência
sonora, quanto mais "confuso" estiver o ouvinte em face daquilo que
o ouve, tanto mais ele sentirá como efetivamente integradas as partes
constitutivas da obra mista; os "dois planos" pressupostamente independentes
e unidos apenas por contingência, aos quais os críticos da música mista
faziam referência, passam a ser percebidos como um \textit{único plano},
essencialmente \textit{diagonal} às linhas estanques da emissão instrumental
ou da difusão puramente eletroacústica; a emissão instrumental passa,
então, a ser efetivamente potencializada no espaço acústico pelos recursos 
eletrônicos. Ainda que de forma alguma hegemônico, o \textit{estado de
dúvida} traduz-se como momento supremo da interação.

O contraste, por sua vez, ancora-se sobretudo na diferença e na 
\textit{distinção absoluta}. Em seus momentos mais acentuados, faz com
que a emissão instrumental ou a eletroacústica assumam o papel estrutural
do silêncio ou, ao contrário, adquiram autonomia temporal e até mesmo
excludente com relação à outra esfera sonora.


copiar mais a página 388 (10.png)

- > usar essa página como exemplo do uso da análise como motor 
composicional
