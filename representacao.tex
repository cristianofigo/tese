\subsubsection{Representação musical}

A análise de áudio conduz para a representação simbólica de 
primeira instância. A primeira segmentação da análise
de áudio, diz respeito ao nível da nota, sua frequência (pitch),
amplitude e ataque. O resultado da análise de áudio é convertido
em um fluxo de eventos MIDI.

Nessa pesquisa, optou-se pelo protocolo MIDI como um intermediário
entre as descrições da análise de áudio e o conjunto de representações
simbólicas envolvidas no sistema. Especialmente os algoritmos de geração
de alturas e amplitudes, além dos harmonizadores.

Muitos autores já discorreram sobre os limites da representação musical
com o protocolo MIDI. O objetivo é usar o MIDI como protocolo de comunicação
com outros programas para manipulação dos algoritmos geradores.
De certa maneira a forte relação do protocolo MIDI com a música instrumental
é um ponto positivo nessa pesquisa que prevê interação musical através de 
instrumentos tradicionais. 
%%original: http://www.lamut.musica.ufrj.br/lamutpgs/rcpesqs/02coper.htm
A reação dos compositores de música eletroacústica em relação ao protocolo MIDI
é bem exposto por Rodolfo Caesar \cite{caesarcopa}:

\begin{quotation}
A música eletroacústica e o protocolo MIDI não foram feitos um para o outro: a especificidade da 
primeira não encontra ressonância imediata no segundo. Se quisermos fazer uso do protocolo MIDI 
para a música eletroacústica, é preciso algum empenho contra suas limitações. 
\end{quotation}


Podemos entender esse tipo de reação, se pensarmos que o MIDI foi um padrão
que percorreu desde os estúdios de música comercial até as pesquisas experimentais
de música interativa nos anos 80. Robert Rowe explica \cite{rowe05}:

\begin{quotation}
 Interactive music systems in early implementations usually made use of the Musical
Instrument Digital Interface (MIDI) standard. The MIDI standard has been recognized since its inception
to be slow and limited in its scope of representation. Reliance on outboard MIDI gear has
doomed a generation of interactive works to obsolescence as the requisite hardware becomes unavailable.
Faster and cheaper machines have in recent years made it possible to perform analysis, synthesis,
sampling, and effects on the CPU of a general purpose personal computer, simultaneously with the
execution of control level software.
\end{quotation}


MIDI é usado na prototipação de algoritmos que dizem respeito as questões de 
altura, duração e dinâmica. De maneira geral, não faz sentido usar MIDI em situações 
musicais que não são pensadas sob o paradigma da nota tocada. O que não é
necessariamente uma regra, pois ainda nesses casos o protocolo MIDI pode ser
útil.

No nível da representação simbólica, os dados são convertidos para MIDI e manipulados
com estruturas híbridas como arrays e listas de números. Apesar de obsoleto, muito desenvolvimento
ainda é feito pensando no protocolo MIDI. O protocolo OSC\footnote{Open Sound Control} permite que
sejam enviados pacotes de dados via rede pelo protocolo UDP/IP. Nesses pacotes podemos incluir
mensagens MIDI inteira ou fragmentadas e reconstruídas na extremidade de quem está recebendo.
O modelo de representação musical no Pd é uma combinação dos principais protocolos disponíveis para
música interativa. Isso permite uma definição híbrida na representação dos dados, como é o caso
nessa pesquisa.

